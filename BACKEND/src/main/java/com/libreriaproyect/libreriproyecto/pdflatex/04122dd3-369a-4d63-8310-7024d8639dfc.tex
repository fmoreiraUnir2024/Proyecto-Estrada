\documentclass{article}
\usepackage{hyperref}
\usepackage{graphicx}

\begin{document}

\begin{center}
\textbf{Cayo o Gayo Julio César}
\end{center}

\begin{justify}
\textbf{Cayo o Gayo Julio César} en latín: Gaius Iulius Caesar; 12 o 13 de julio de 100 a. C.-15 de marzo de 44 a. C. fue un político y militar romano del siglo i a. C., miembro de los patricios Julios Césares, que alcanzó las más altas magistraturas del Estado romano y dominó la política de la República tras vencer en la guerra civil que le enfrentó al sector más conservador del Senado. Nacido en el seno de la gens Julia, una familia patricia de escasa fortuna, estuvo emparentado con algunos de los hombres más influyentes de su época, como su tío Cayo Mario, quien influiría de manera determinante en su carrera política. En 84 a. C., a los 16 años, el popular Lucio Cornelio Cinna lo nombró flamen Dialis, cargo religioso del que fue relevado por Sila, con quien tuvo conflictos a causa de su matrimonio con la hija de Cinna. Tras escapar de morir a manos de los sicarios del dictador Sila, fue perdonado gracias a la intercesión de los parientes de su madre. Trasladado a la provincia de Asia, combatió en Mitilene como legado de Marco Minucio Termo. Volvió a Roma a la muerte de Sila en 78 a. C., y ejerció por un tiempo la abogacía. En 73 a. C. sucedió a Cayo Aurelio Cota como pontífice, y pronto entró en relación con los cónsules Pompeyo y Craso, cuya amistad le permitiría lanzar su propia carrera política. En 70 a. C. César sirvió como cuestor en la provincia de la Hispania Ulterior y como edil curul en Roma. Durante el desempeño de esa magistratura ofreció unos espectáculos que fueron recordados durante mucho tiempo por el pueblo.
\end{justify}

\end{document}